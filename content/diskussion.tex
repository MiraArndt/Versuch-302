\section{Diskussion}
\label{sec:Diskussion}

Es ist auffällig, dass einige Messungen sehr präzise Werte für die Eigenschaften der Bauelemente liefern, wie 
bespielsweise in \ref{WheatstonescheBrücke}, wohingegen andere Messungen, wie zum Bespiel \ref{Kapazitätsmessbrücke} sehr 
hohe Standartabweichungen der Mittelwerte aufweisen.

Ganz besonders sticht dies bei dem Vergleich der errechneten Werte von \ref{Induktivitätsmessbrücke} und \ref{MaxwellBrücke} hervor.
Diese sollten zwar dieselben Werte liefern, allerdings erscheint lediglich der in \ref{MaxwellBrücke} ermittelte Wert für den 
ohmschen Widerstand der RC-Kombination sinnvoll zu sein, da der in \ref{Induktivitätsmessbrücke} einen viel zu großen Fehler der 
Standartabweichung ausweist. Der in \ref{Induktivitätsmessbrücke} errechnete Wert lässt sich nicht durch übliche systematische beziehungsweise
statistische Fehler erklären. Da mit Ausnahme von $L_{2}$ in beiden Versuchen die gleichen Bauelemente verwendet wurden, liegt 
die Vermutung nahe, dass sich der Fehler in der Berechnung auf eine fehlerhaft funktionierende Spule zurückführen lässt.

Das die experimentell gemessene Frequenz $\nu_{0}$ fast mit dem in der Theorie berechneten Wert übereinstimmt, kann auf den 
geringen Klirrfaktor $k$ zurückgeführt werden, da das Spannungsminimum dadurch gut zu erkennen war. Allerdings wurde bei der Berechnung
des Klirrfaktors auch die Näherung vorgenommen, dass alle Oberwellen, mit Außnahme der zweiten, vernachlässigt wurden.
Der größer werdende Abstand der Messwerte  von der Theoriekurve in Abbildung \ref{fig:plot}, 
für Frequenzverhältnisse, die von der dem Wert $\Omega = 1$ abweichen, lässt sich möglicherweise damit erklären, dass die Skala für 
die Frequenz der Wechselspannung an der Spannungsquelle vor allem für größere Frequenzen ungenauer wird. Folglich würde dies einen 
systematische Fehler hervorrufen, der mit zunehmenden Abstand von $\nu_{0}$ selbst zunimmt.
Allgemein könnten Unterschiede von den errechneten und den tatsächlichen Größen der Bauelemente dadurch zustande kommen, dass, wie zuvor erwähnt,
keine baubedingten Fehler beachtet wurden, aber auch, dass davon ausgegangen wurde, dass es sich bei den Kondensatoren und Spulen um ideale 
Bauelemente handelt, sprich das diese vollkommen verlustfrei sind.
