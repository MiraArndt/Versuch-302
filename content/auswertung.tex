\section{Auswertung}
\label{sec:Auswertung}

Bei der Berechnung der jeweiligen Größen wurd
ggf Wert nennen


\subsection{Wheatstonesche Brücke}

Mit denen verwendeten Widerständen, die in Tabelle \ref{tab:1} aufgeführt wurden, 
lassen sich durch Gleichung (VERWEIS AUF GLEICHUNG) folgende Werte für den 
unbekannten Widerstandswert $R_{x}$ berechnen:
(Fehlerhafter AUSDRUCK)

\begin{align}
R_{x,1} = 491,821\,\upOmega \nonumber \\
R_{x,2} = 492,794\,\upOmega \nonumber \\
R_{x,3} = 490,313\,\upOmega \nonumber 
\end{align}.



\noindent
Über die zuvor aufgeführte Gleichungen (VERWEIS AUF GLEICHUNGEN) lässt sich der  
Mittelwert 

\begin{equation}
\bar{R_{x}} = 491,643\,\upOmega \nonumber
\end{equation}, 

\noindent
samt zugehörigem Fehler der Standartabweichung

\begin{equation}
\Delta\bar{R} = 0,722\, \upOmega \nonumber
\end{equation}

\noindent 
ermitteln.

\noindent
Das zusammengefasste Ergebnis für den, mithilfe der Wheatstonesche Brückenspannung berechneten, Widerstandswert
lautet, wie folgt:

\begin{equation}
R_{x} = (491,643 \pm 0,722)\,  \upOmega \nonumber
\end{equation}.
    
\begin{table}
\normalsize

\centering
\sisetup{table-format=4.0}
\begin{tabular}{c c c c}
\toprule
        Messung & $R_{2} \,/\,\upOmega$ & $R_{3} \,/\,\upOmega$ & $R_{4} \,/\,\upOmega$ \\
        \midrule
        1 & 332 & 597 & 403  \\
        2 & 664 & 426 & 574  \\
        3 & 1000 & 329 & 671 \\ 

\bottomrule

\end{tabular}

\caption{Text}
\label{tab:1}
\end{table}



\subsection{Kapazitätsmessbrücke}

Unter Verwendung der oben ausgeführten Gleichung (BEZUG AUF GLEICHUNG) sowie der aufgenommenen
Messwerte aus Tabelle \ref{tab2} können die Werte 
\begin{align}
R_{15,1} = 538.899\,\upOmega \nonumber \\
R_{15,2} = 474.937\,\upOmega \nonumber \\ 
\end{align}
\\
für den ohmschen Widerstand in der RC-Kombination Nummer 15 und
\begin{align}
C_{15,1} = \num{491.625e-9}\,\upOmega \nonumber \\
C_{15,2} = \num{629.986e-9}\,\upOmega \nonumber \\ 
\end{align}
\\
für die Kapazitäten in der RC-Kombination ermittelt werden.






bei den Messungen 1 und 2 


\begin{table}
\normalsize

\centering
\sisetup{table-format=4.0}
\begin{tabular}{c c c c c}
\toprule
        Messung & $R_{2} \,/\,\upOmega$ & $R_{3} \,/\,\upOmega$ & $R_{4} \,/\,\upOmega$ & $C_{2} \,/\, \si{\farad}$ \\
        
        \midrule
        1 & 664 & 448 & 552 & \num{399e-9} \\
        2 & 664 & 417 & 583 & \num{450e-9} \\

\bottomrule

\end{tabular}

\caption{Text2 WERT 15}
\label{tab:2}
\end{table}


\begin{table}
\normalsize

\centering
\sisetup{table-format=4.0}
\begin{tabular}{c c c c c}
\toprule
        Messung & $R_{2} \,/\,\upOmega$ & $R_{3} \,/\,\upOmega$ & $R_{4} \,/\,\upOmega$ & $C_{2} \,/\, \si{\farad}$ \\
        
        \midrule
        1 & 332 & 419 & 581 & \num{399e-9} \\

\bottomrule

\end{tabular}

\caption{Text2 WERT 3 (C) und WERT 10 (R)} 
\label{tab:3}
\end{table}





