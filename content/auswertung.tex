\section{Auswertung}
\label{sec:Auswertung}

Im Folgenden wurden die baubedingten Fehler sämtlicher Bauteile vernachlässigt und 
treten somit auch nicht in den Fehlerrechnungen auf. Diese beschränken sich lediglich auf 
die Berechnung der Mittelwerte, sowie die damit verbundenen Fehler der Standartabweichungen.


\subsection{Wheatstonesche Brücke}

Mit denen verwendeten Widerständen, die in Tabelle \ref{tab:1} aufgeführt wurden, 
lassen sich durch Gleichung (VERWEIS AUF GLEICHUNG) folgende Werte für den 
unbekannten Widerstandswert $R_{x}$ berechnen:
(Fehlerhafter AUSDRUCK)

\begin{align}
R_{x,1} = 491,821\,\si{\ohm} \nonumber \\
R_{x,2} = 492,794\,\si{\ohm} \nonumber \\
R_{x,3} = 490,313\,\si{\ohm} \nonumber 
\end{align}.

\noindent
Über die zuvor aufgeführte Gleichungen (VERWEIS AUF GLEICHUNGEN) lässt sich der  
Mittelwert 

\begin{equation}
\bar{R_{x}} = 491,643\,\si{\ohm} \nonumber
\end{equation}, 

\noindent
samt zugehörigem Fehler der Standartabweichung

\begin{equation}
\Delta R_{x} = 0,722\, \si{\ohm} \nonumber
\end{equation}

\noindent 
ermitteln.

\noindent
Das zusammengefasste Ergebnis für den, mithilfe der Wheatstonesche Brückenspannung berechneten, Widerstandswert
lautet demnach

\begin{equation}
R_{x} = (491,643 \pm 0,722)\,  \si{\ohm} \nonumber
\end{equation}.
    
\begin{table}
\normalsize

\centering
\sisetup{table-format=4.0}
\begin{tabular}{c c c c}
\toprule
        Messung & $R_{2} \,/\,\si{\ohm}$ & $R_{3} \,/\,\si{\ohm}$ & $R_{4} \,/\,\si{\ohm}$ \\
        \midrule
        1 & 332 & 597 & 403  \\
        2 & 664 & 426 & 574  \\
        3 & 1000 & 329 & 671 \\ 

\bottomrule

\end{tabular}

\caption{Text}
\label{tab:1}
\end{table}

\subsection{Kapazitätsmessbrücke}

Unter Verwendung der oben ausgeführten Gleichung (BEZUG AUF GLEICHUNG) sowie der aufgenommenen
Messwerte aus Tabelle \ref{tab2} können die Werte 
\begin{align}
R_{15,1} = 538.899\,\si{\ohm} \nonumber \\
R_{15,2} = 474.937\,\si{\ohm} \nonumber  
\end{align}
\\
für den ohmschen Widerstand und
\begin{align}
C_{15,1} = 491.625\,\si{\nano\ohm} \nonumber \\
C_{15,2} = 629.986\,\si{\nano\ohm} \nonumber 
\end{align}
\\
für die Kapazitäten in der RC-Kombination Nummer 15 ermittelt werden. Mithilfe der Gleichung (VERWEIS AUF GLEICHUNG)
lassen sich 
\begin{align}
R_{15} = (506.918 \pm 50.566)\, \si{\ohm} \nonumber 
\end{align}
\\
und
\begin{align}
C_{15} = (560.806 \pm 67.181)\, \si{\nano\farad} \nonumber 
\end{align}
\\
als Mittelwerte samt Standartabweichungen für den ohmschen Widerstand beziehungsweise der
Kapazität der RC-Kombination Nummer 15 benennen.

Im Folgenden setzt sich die RC-Kombination aus dem Kondensator Nummer 3 und dem Widerstand Nummer 10 zusammen. Weiterhin 
können die in Tabelle (VERWEIS AUF TABELLE) aufgeführten Messwerte verwendet werden, um über Gleichung
(VERWEIS AUF GLEICHUNG) 
\begin{align}
\bar{R_{10,1}} = 239.429\, \si{\ohm} \nonumber
\end{align}
\\
als ohmscher Widerstand zu Bauteil Nummer 10 und
\begin{align}
\bar{C_{3,1}} = 553.267\, \si{\nano\farad} \nonumber
\end{align}
\\
als Kapazität des Bauteils Nummer 3 zu identifizieren. Da nur eine Messung durchgeführt wurde, können lediglich $\bar{R_{10,1}}$
und $\bar{C_{3,1}}$ angegeben werden, nicht aber Mittelwerte beziehungsweise Fehler der Standartabweichungen.


\begin{table}
\normalsize

\centering
\sisetup{table-format=4.0}
\begin{tabular}{c c c c c}
\toprule
        Messung & $R_{2} \,/\,\si{\ohm}$ & $R_{3} \,/\,\si{\ohm}$ & $R_{4} \,/\,\si{\ohm}$ & $C_{2} \,/\, \si{\farad}$ \\
        
        \midrule
        1 & 664 & 448 & 552 & \num{399e-9} \\
        2 & 664 & 417 & 583 & \num{450e-9} \\

\bottomrule

\end{tabular}

\caption{Text2 WERT 15}
\label{tab:2}
\end{table}


\begin{table}
\normalsize

\centering
\sisetup{table-format=4.0}
\begin{tabular}{c c c c c}
\toprule
        Messung & $R_{2} \,/\,\si{\ohm}$ & $R_{3} \,/\,\si{\ohm}$ & $R_{4} \,/\,\si{\ohm}$ & $C_{2} \,/\, \si{\farad}$ \\
        
        \midrule
        1 & 332 & 419 & 581 & \num{399e-9} \\

\bottomrule

\end{tabular}

\caption{Text2 WERT 3 (C) und WERT 10 (R)} 
\label{tab:3}
\end{table}

\subsection{Induktivitätsmessbrücke}

Für diesen Teil des Versuchs können die Werte aus Tabelle (VERWEIS AUF TABELLE) und die Gleichung
(VERWEIS AUF GLEICHUNG) verwendet werden, sodass die Ergbenisse der Einzelmessungen

\begin{align}
R_{18,1} = 3184.100\, \si{\ohm} \nonumber \\
R_{18,2} = 1130.555\, \si{\ohm} \nonumber \\
R_{18,3} = 2114.243\, \si{\ohm} \nonumber 
\end{align}
\\
für den ohmschen Widerstand $R_{18}$ und
\begin{align}
L_{18,1} = 46.448\, \si{\milli\henry} \nonumber \\
L_{18,2} = 49.717\, \si{\milli\henry} \nonumber \\
L_{18,3} = 46.488\, \si{\milli\henry} \nonumber 
\end{align} 
\\
für die Induktivität $L_{18}$ der LR-Kombination liefern. Unter der Zuhilfenahme von Gleichung 
(VERWEIS AUF GLEICHUNG) lassen sich $R_{18}$ und $L_{18}$ durch die errechneten Werte 
\begin{align}
R_{18} = (2142.966 \pm 592.981)\, \si{\ohm} \nonumber \\
L_{18} = (47.564 \pm 1.076)\, \si{\henry} \nonumber 
\end{align}
\\ 
angeben.

\begin{table}
\normalsize

\centering
\sisetup{table-format=4.0}
\begin{tabular}{c c c c c}
\toprule
        Messung & $R_{2} \,/\,\si{\ohm}$ & $R_{3} \,/\,\si{\ohm}$ & $R_{4} \,/\,\si{\ohm}$ & $L_{2} \,/\, \si{\henry}$ \\
        \midrule
        1 & 1000 & 761 & 239 & \num{14.6e-3} \\
        2 & 332 & 773 & 227 & \num{14.6e-3} \\
        3 & 664 & 761 & 239 & \num{14.6e-3} \\
\bottomrule
\end{tabular}
\caption{Text4} 
\label{tab:4}
\end{table}

\subsection{Maxwell-Brücke}

Um den ohmschen Widerstand $R_{18}$, sowie die Induktivität $L_{18}$, der LR-Kombination ein weiteres zu errechnen, sollen
nun die Werte aus Tabelle (VERWEIS AUF TABELLE) und die beiden Gleichungen (VERWEIS AUF GLEICHUNGEN) verwendet werden. Somit
ergeben sich für $R_{18}$
\begin{align}
R_{18,1} = 208.000\, \si{\ohm} \nonumber \\
R_{18,2} = 204.000\, \si{\ohm} \nonumber \\
R_{18,3} = 204.819\, \si{\ohm} \nonumber 
\end{align}.
\\Ein analoges Vorgehen ergibt
\begin{align}
L_{18,1} = \num{51.792e-3}\, \si{\henry} \nonumber \\
L_{18,2} = \num{50.796e-3}\, \si{\henry} \nonumber \\
L_{18,3} = \num{51,000}\, \si{\milli\henry} \nonumber 
\end{align}
\\ 
als Werte für $L_{18}$. Daran geschlossen können die beiden gesuchten Größen unter Verwendung von Gleichung a  b
(VERWEIS AUF GLEICHUNG)
über die Mittelwerte der Messungen, sowie den Fehler der Standartabweichung angegeben werden. Folglich ergibt sich
\begin{align}
R_{18} = (205.606 \pm 1.220)\, \si{\ohm} \nonumber
\end{align}
\\
für den ohmschen Widerstand $R_{18}$ und
\begin{align}
L_{18} = (51.196 \pm 0.304)\, \si{\milli\henry} \nonumber
\end{align}
\\
für die Induktivität $L_{18}$ der LR-Kombination.

\begin{table}
\normalsize
\centering
\sisetup{table-format=4.0}
\begin{tabular}{c c c c c}
\toprule
        Messung & $R_{2} \,/\,\si{\ohm}$ & $R_{3} \,/\,\si{\ohm}$ & $R_{4} \,/\,\si{\ohm}$ & $C_{4} \,/\, \si{\farad}$ \\
        \midrule
        1 & 332 & 208 & 332 & \num{750e-9} \\
        2 & 664 & 102 & 332 & \num{750e-9} \\
        3 & 1000 & 68 & 32 & \num{750e-9} \\
\bottomrule
\end{tabular}
\caption{Text5} 
\label{tab:5}
\end{table}

\subsection{Frequenzabhängigkeit der Brückenspannung einer Wien-Robinsson-Brücke}

Um den Theoriewert für $\nu_{0}$ zu erhalten, muss zunächst $\omega_{0}$ mit 
\begin{align}
\omega_{0} = \frac{1}{R \cdot C} \nonumber
\end{align}
\\ 
berechnet werden. Durch Einsetzen der Größen, die in der Tabelle (VERWEIS AUF TABELLE) 
aufgeführt sind, ergibt sich
\begin{align}
\omega_{0} = \frac{1}{1000\, \si{\ohm} \cdot 420\, \si{\nano\farad}} = 2380.952\, \si{\hertz}
\end{align}
\\ 
als Kreisfrequenz. Nach Umrechnung der Kreisfrequenz $\omega_{0}$ in die Frequenz $\nu_{0}$ mithilfe von 
\begin{equation}
\nu_{0} = \frac{\omega_{0}}{2 \cdot \pi}
\end{equation}
\\
ergibt sich der Theoriewert 
\begin{align}
\nu_{0} = \frac{2380.952\, \si{\hertz}}{2 \cdot \pi} = 378.94\, \si{\hertz}
\end{align}
für die Kreisfrequenz, bei der die minimale Brückenspannung $U_{Br}$ gemessen werden kann.

In Abbildung (VERWEIS AUF DIE ABBILDUNG) wurden die Messwerte, ebenso wie die mit Gleichung (VERWEIS AUF GLEICHUNG) 
berechneten Werte für die Theoriekurve, aufgetragen. Die x-Achse ist das Verhältnis $\upOmega$ von $\nu$ zu $\nu_{0}$ 
logarithmisch aufgefragen, wohingegen die y-Achse das Verhältnis von der Brückenspannung $U_{Br}$ zu der Speisespannung
$U_{S}$ widergibt.

\noindent
\begin{figure}
  \centering
  \includegraphics{plot.pdf}
   \caption{TITEL}
   \label{fig:plot}
\end{figure}

\begin{table}
\normalsize
\centering
\sisetup{table-format=4.0}
\begin{tabular}{c c c c}
\toprule
        $2R' \,/\,\si{\ohm}$ & $R' \,/\,\si{\ohm}$ & $R \,/\,\si{\ohm}$ & $C_{4} \,/\, \si{\farad}$ \\
        \midrule
        664 & 332 & 1000 & \num{420e-9} \\
\bottomrule
\end{tabular}
\caption{Text5} 
\label{tab:6}
\end{table}

\noindent
\begin{table}
\normalsize
\centering
\sisetup{table-format=4.0}
\begin{tabular}{c c c c}
\toprule
        $U_{S} \,/\,\si{\milli\volt}$ & $U_{Br} \,/\,\si{\milli\volt}$ & $\upOmega$ & $\nu \,/\, \si{\henry}$ \\
        \midrule
        2500 & 1320 &           0.0789 & 30 \\
        2500 & 1200 &           0.2105 & 80 \\
        2500 & 880 &            0.3221 & 130 \\
        2500 & 640 &            0.4737 & 180 \\
        2500 & 460 &            0.6053 & 230 \\
        2500 & 268 &            0.7368 & 280 \\
        2500 & 128 &            0.8684 & 330 \\
        2500 & 94.4 &           0.8947 & 340 \\
        2500 & 70.4 &           0.9211 & 350 \\
        2500 & 44.0 &           0.9474 & 360 \\
        2500 & 21.6 &           0.9737 & 370 \\
        2500 & 13.6 &           1.0000 & 380 \\   
        2500 & 30 &             1.0263 & 390 \\
        2500 & 52 &             1.0526 & 400 \\
        2500 & 78 &             1.0789 & 410 \\
        2500 & 96 &             1.1053 & 420 \\
        2500 & 118 &            1.1316 & 430 \\
        2500 & 208 &            1.2631 & 480 \\
        2500 & 296 &            1.3947 & 530 \\
        2500 & 400 &            1.5263 & 580 \\
        2500 & 472 &            1.6579 & 630 \\
        2500 & 536 &            1.7894 & 680 \\
        2500 & 584 &            1.9210 & 730 \\
        2500 & 640 &            2.0526 & 780 \\
        
\bottomrule
\end{tabular}
\caption{Text5} 
\label{tab:7}
\end{table}

\subsection{Klirrfaktormessung}

Bevor der Klirrfaktor $k$ ermittelt werden kann, muss zunächst die Amplitude $U_{2}$ der 2-ten Oberwelle
mit der Gleichung
\begin{align}
U_{2} = \frac{U_{Br,eff}}{f(2)}
\end{align}
\\
berechnet werden, wobei sich $f(2)$ ergibt, indem $\upOmega = 2$ in Gleichung (VERWEIS AUF GLEICHUNG) eingesetzt wird.
Neben dem dort errechneten Wert
\begin{align}
f(2) =  \frac{1}{9} \cdot \frac{(\Omega^2 - 1)^2}{(1 - \Omega^2)^2 + 9 \cdot \Omega^2}  = \frac{\sqrt{5}}{15} = \frac{1}{\sqrt{45}} \nonumber
\end{align}
\\
wird der Effektivwert der Wechselspannung $U_{Br}$ bei $\nu = \nu_{0}$ benötigt (von der Speisespannung ist bereits der Effektivwert gegeben). Dieser lautet
\begin{align}
U_{Br,eff} = \frac{U_{Br}}{\sqrt{2}} = 9.6166\, \si{\milli\volt} \nonumber
\end{align}.
\\
Folglich lautet der Wert für die Amplitude der 2-ten Oberwelle
\begin{align}
U_{2} = \frac{9.6166\, \si{\milli\volt}}{\sqrt{\frac{1}{45}}} = 0.0645\, \si{\volt} \nonumber
\end{align}
\\womit sich als Kliffaktor $k$ der Wert
\begin{align}
k = \frac{U_{2}}{U_{1}} = \frac{0.0645\, \si{\volt}}{2.5\, \si{\volt}} = 0.0258 \nonumber
\end{align}
\\ 
ergibt. Die Amplitude der Grundwelle entspricht hierbei dem Effektivwert der Speisespannung.
