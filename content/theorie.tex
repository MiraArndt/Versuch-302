\section{Theorie}
\label{sec:Theorie}
Brückenschaltungen werden in der Messtechnik 
eingesetzt um die Auflösung einer Messung zu erhöhen
oder eine pysikalische Größe,
die sich als elektrischer Widerstand darstellen lässt,
zu bestimmen. \\
\\
Dafür muss eine Abgleichbedingung der
Brückenschaltung erfüllt sein. Generell benötigt eine
Brückenschaltung eine Speisespannung $U_s$, den zu
ermittelnden elektrischen Widerstand und bekannte
elektrische Bauteile um ein Widerstandsverhältnis
zu bestimmen.
Die Abgleichbedingung besteht darin, dass die
Brückenspannung $U_Br$ zwischen zwei Punkten
verschwindet.

\begin{figure}[H]
\centering
    \includegraphics[height= 5cm]{content/Allgemein.png}
\end{figure}

\noindent Ist die Abgleichbedingung erfüllt kann
aus dem Widerstandsverhältnis der
unbekannte Widerstand bestimmt werden.\\
\\
Dieses Verhältnis ergibt sich aus den beiden
Kirchhoffschen Gesetzen
\begin{equation}
   \sum_{k} I_k =0
\end{equation}
\begin{equation}
    \sum_{k} U_k =0,
\end{equation}

\noindent die besagen, dass die Summe aller eingehenden
Ströme eienes Knotens gleich der Summe aller ausgehenden
Ströme ist und die Summe aller Spannungen in einer Masche
immer Null ist.\\
\\

Sobald $U_Br$




\subsection{Wheatstonesche Brücke}
\subsection{Kapazitätsmessbrücke}
\subsection{Induktivitätsmessbrücke}
\subsection{Maxwell-Brücke}
\subsection{Wien-Robinson-Brücke}
\subsection{Fehlerrechnung}
Bei der Auswertung werden die Mittelwerte 
der errechneten Größen durch die Formel
\begin{equation}
    \bar{x}=\frac{1}{N}\sum_{i=1}^N x_i
\end{equation}
berechnet.\\ 
\\
Der Standardfehler des Mittelwerts beerechnet sich durch

 \begin{equation}
     \Delta\bar{x}=\sqrt{\frac{1}{N(N-1)}\sum_{i=1}^N (x_i-\bar{x})}.
 \end{equation}
