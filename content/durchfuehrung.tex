\section{Durchführung}
\label{sec:Durchführung}

Die Schaltungen werden jeweils wie auf den Schaltbildern bei 
\ref{sec:Theorie} aufgebaut. Dabei beträgt die
Speisespannung mit Ausnahme der Messung
zur Wien-Robinson-Brücke 1\,V bei einer Frequenz von 20\,Hz. Die Brückenspannung wird
mit einem Oszilloskop gemessen.
\subsection{Wheatstonesche Brücke}
Der unbekannte Widerstand ist der Ohmsche Widerstand Wert 11.
Es werden drei Messungen durchgeführt bei denen jeweils Der Widerstand $R_2$
variiert wird. Das Potentiometer wird so eingestellt,
dass die Brückenspannung verschwindet und die Werte
für $R_3$ und $R_4$ werden zusätzlich zu $R_2$ festgehalten.
\subsection{Kapazitätsmessbrücke}
Die unbekannte Kapazität ist teil einer RC-Kombination,
bei der direkt auch der unbekannte Ohmsche Widerstand realisiert ist.
Bei den ersten beiden Messungen wird die bekannte Kapazität $C_2$ variiert
und bei der dritten Messung eine andere unbekannte Kapazität mit Wert 3
und ein anderer unbekannter ohmscher Widerstand mit Wert 10 gemessen.
Auch hier wird wie oben das Potentiometer passend eingestellt und die
entsprechenden Werte notiert.
\subsection{Induktivitätsmessbrücke}
\label{sec:Indu}
Hier ist die unbekannte Induktivität teil einer LR-Kombination
mit Wert 18. Es werden drei Messungen mit jeweils anderen Widerständen $R_2$
durchgeführt indem wieder das Potentiometer eingestellt und die Werte aufgenommen werden.
\subsection{Maxwell-Brücke}
Es wird die gleiche LR-Kombination wie bei \ref{sec:Indu} vermessen.
Es erfolgen wieder drei Messungen mit variiertem $R_2$. Diesmal
wird jedoch nur $R_3$ durch das Potentiometer angepasst bis die
Brückenspannung verschwindet und der Wert festgehalten.

\subsection{Wien-Robinson-Brücke}
Bei diesem Aufbau werden die elektrischen Bauteile
nicht ausgewechselt sondern nur die Frequenz am 
Generator variiert. Zunächst wird die Frequenz $\nu_0$ eingestellt
bei der die Brückenspannung minimal wird. Um diesem Bereich werden
10 Messungen durchgeführt bei denen die Frequenz jeweils um 10 Hz
variiert wird. Weiter entfernt vom Minimum werden weitere 14 Messungen
mit Frequenzabständen von 50 Hz vorgenommen.

